\documentclass[]{article}
\usepackage{lmodern}
\usepackage{amssymb,amsmath}
\usepackage{ifxetex,ifluatex}
\usepackage{fixltx2e} % provides \textsubscript
\ifnum 0\ifxetex 1\fi\ifluatex 1\fi=0 % if pdftex
  \usepackage[T1]{fontenc}
  \usepackage[utf8]{inputenc}
\else % if luatex or xelatex
  \ifxetex
    \usepackage{mathspec}
  \else
    \usepackage{fontspec}
  \fi
  \defaultfontfeatures{Ligatures=TeX,Scale=MatchLowercase}
\fi
% use upquote if available, for straight quotes in verbatim environments
\IfFileExists{upquote.sty}{\usepackage{upquote}}{}
% use microtype if available
\IfFileExists{microtype.sty}{%
\usepackage{microtype}
\UseMicrotypeSet[protrusion]{basicmath} % disable protrusion for tt fonts
}{}
\usepackage[margin=1in]{geometry}
\usepackage{hyperref}
\hypersetup{unicode=true,
            pdftitle={Tabu Search},
            pdfborder={0 0 0},
            breaklinks=true}
\urlstyle{same}  % don't use monospace font for urls
\usepackage{graphicx,grffile}
\makeatletter
\def\maxwidth{\ifdim\Gin@nat@width>\linewidth\linewidth\else\Gin@nat@width\fi}
\def\maxheight{\ifdim\Gin@nat@height>\textheight\textheight\else\Gin@nat@height\fi}
\makeatother
% Scale images if necessary, so that they will not overflow the page
% margins by default, and it is still possible to overwrite the defaults
% using explicit options in \includegraphics[width, height, ...]{}
\setkeys{Gin}{width=\maxwidth,height=\maxheight,keepaspectratio}
\IfFileExists{parskip.sty}{%
\usepackage{parskip}
}{% else
\setlength{\parindent}{0pt}
\setlength{\parskip}{6pt plus 2pt minus 1pt}
}
\setlength{\emergencystretch}{3em}  % prevent overfull lines
\providecommand{\tightlist}{%
  \setlength{\itemsep}{0pt}\setlength{\parskip}{0pt}}
\setcounter{secnumdepth}{0}
% Redefines (sub)paragraphs to behave more like sections
\ifx\paragraph\undefined\else
\let\oldparagraph\paragraph
\renewcommand{\paragraph}[1]{\oldparagraph{#1}\mbox{}}
\fi
\ifx\subparagraph\undefined\else
\let\oldsubparagraph\subparagraph
\renewcommand{\subparagraph}[1]{\oldsubparagraph{#1}\mbox{}}
\fi

%%% Use protect on footnotes to avoid problems with footnotes in titles
\let\rmarkdownfootnote\footnote%
\def\footnote{\protect\rmarkdownfootnote}

%%% Change title format to be more compact
\usepackage{titling}

% Create subtitle command for use in maketitle
\newcommand{\subtitle}[1]{
  \posttitle{
    \begin{center}\large#1\end{center}
    }
}

\setlength{\droptitle}{-2em}

  \title{Tabu Search}
    \pretitle{\vspace{\droptitle}\centering\huge}
  \posttitle{\par}
    \author{}
    \preauthor{}\postauthor{}
    \date{}
    \predate{}\postdate{}
  

\begin{document}
\maketitle

\documentclass[]{article}
\usepackage{lmodern}
\usepackage{amssymb,amsmath}
\usepackage{ifxetex,ifluatex}
\usepackage{fixltx2e} % provides \textsubscript
\ifnum 0\ifxetex 1\fi\ifluatex 1\fi=0 % if pdftex
  \usepackage[T1]{fontenc}
  \usepackage[utf8]{inputenc}
\else % if luatex or xelatex
  \ifxetex
    \usepackage{mathspec}
  \else
    \usepackage{fontspec}
  \fi
  \defaultfontfeatures{Ligatures=TeX,Scale=MatchLowercase}
\fi
% use upquote if available, for straight quotes in verbatim environments
\IfFileExists{upquote.sty}{\usepackage{upquote}}{}
% use microtype if available
\IfFileExists{microtype.sty}{%
\usepackage{microtype}
\UseMicrotypeSet[protrusion]{basicmath} % disable protrusion for tt fonts
}{}
\usepackage[margin=1in]{geometry}
\usepackage{hyperref}
\hypersetup{unicode=true,
            pdftitle={Tabu Search},
            pdfborder={0 0 0},
            breaklinks=true}
\urlstyle{same}  % don't use monospace font for urls
\usepackage{graphicx,grffile}
\makeatletter
\def\maxwidth{\ifdim\Gin@nat@width>\linewidth\linewidth\else\Gin@nat@width\fi}
\def\maxheight{\ifdim\Gin@nat@height>\textheight\textheight\else\Gin@nat@height\fi}
\makeatother
% Scale images if necessary, so that they will not overflow the page
% margins by default, and it is still possible to overwrite the defaults
% using explicit options in \includegraphics[width, height, ...]{}
\setkeys{Gin}{width=\maxwidth,height=\maxheight,keepaspectratio}
\IfFileExists{parskip.sty}{%
\usepackage{parskip}
}{% else
\setlength{\parindent}{0pt}
\setlength{\parskip}{6pt plus 2pt minus 1pt}
}
\setlength{\emergencystretch}{3em}  % prevent overfull lines
\providecommand{\tightlist}{%
  \setlength{\itemsep}{0pt}\setlength{\parskip}{0pt}}
\setcounter{secnumdepth}{0}
% Redefines (sub)paragraphs to behave more like sections
\ifx\paragraph\undefined\else
\let\oldparagraph\paragraph
\renewcommand{\paragraph}[1]{\oldparagraph{#1}\mbox{}}
\fi
\ifx\subparagraph\undefined\else
\let\oldsubparagraph\subparagraph
\renewcommand{\subparagraph}[1]{\oldsubparagraph{#1}\mbox{}}
\fi

%%% Use protect on footnotes to avoid problems with footnotes in titles
\let\rmarkdownfootnote\footnote%
\def\footnote{\protect\rmarkdownfootnote}

%%% Change title format to be more compact
\usepackage{titling}

% Create subtitle command for use in maketitle
\newcommand{\subtitle}[1]{
  \posttitle{
    \begin{center}\large#1\end{center}
    }
}

\setlength{\droptitle}{-2em}

  \title{Tabu Search}
    \pretitle{\vspace{\droptitle}\centering\huge}
  \posttitle{\par}
    \author{}
    \preauthor{}\postauthor{}
    \date{}
    \predate{}\postdate{}
  

\begin{document}
\maketitle

\subsection{R Markdown}\label{r-markdown}

\section{Loss Function}\label{loss-function}

L = function(A, B, C, t\_c, m, omega, phi, price, t)\{ loss \textless{}-
sum((price - (A + B\emph{abs(t\_c - t)\^{}m +
(C}abs(t\_c-t)\^{}m\emph{cos(omega } log(abs(t\_c - t)) - phi))))\^{}2)
\}

TabuSearch \textless{}- function(A, B, C)\{ S \textless{}- hash() \# S
represents current solution iter\_without\_improvement \textless{}- 0

\# Generating a set of initial 10 * 3 parameters t\_c\_set \textless{}-
runif(n = 40, min = max(base\_data\(t), max = max(base_data\)t) + 0.6)
\# adding 0.6 year represent time horizon within the crash is expected
m\_set \textless{}- runif(n = 40, min = 0.1, max = 0.9) omega\_set
\textless{}- runif(n = 40, min = 6, max = 13) phi\_set \textless{}-
runif(n = 40, min = 0, max = 2*pi)

\# Looking for the 10 elite solutions out of 30 initial points, the best
one is then our starting point. Furthermore, we choose the worst
solution to set the taboo condition. elite\_list \textless{}- vector()
random\_solutions \textless{}- cbind(t\_c\_set, m\_set, omega\_set,
phi\_set) losses \textless{}- apply(random\_solutions, 1, function(x)
L(A = A, B = B, C = C, t\_c = x{[}1{]}, m = x{[}2{]}, omega = x{[}3{]},
phi = x{[}4{]}, price = base\_data\(price, t = base_data\)t)) losses
\textless{}- as.data.frame(cbind(`loss' = losses, `t\_c' = t\_c\_set,
`m' = m\_set, `omega' = omega\_set, `phi' = phi\_set)) losses
\textless{}- losses \%\textgreater{}\% arrange(loss) elite\_list
\textless{}- losses{[}1:10,{]} taboo\_condition \textless{}-
losses{[}nrow(losses),1{]} S{[}{[}`loss'{]}{]} \textless{}-
losses{[}1,`loss'{]} S{[}{[}`t\_c'{]}{]} \textless{}-
losses{[}1,`t\_c'{]} S{[}{[}`m'{]}{]} \textless{}- losses{[}1,`m'{]}
S{[}{[}`omega'{]}{]} \textless{}- losses{[}1,`omega'{]}
S{[}{[}`phi'{]}{]} \textless{}- losses{[}1,`phi'{]} if (min(losses,
na.rm = TRUE) \textless{} 200)\{ \# Partitioning and setting parameters
for the number of randomly drawn cells and points within them partitions
\textless{}- c(6,6,6,6) n\_c \textless{}- 2 n\_s \textless{}- 6
t\_c.partitions \textless{}- seq(from =
max(base\_data\(t), to = max(base_data\)t) + 4, length.out =
partitions{[}1{]} + 1) \# we add 1 to create 6 cells, which requires 7
borders m.partitions \textless{}- seq(from = 0.1, to = 0.9, length.out =
partitions{[}2{]} + 1) omega.partitions \textless{}- seq(from = 6, to =
13, length.out = partitions{[}3{]} + 1) phi.partitions \textless{}-
seq(from = 0, to = 2*pi, length.out = partitions{[}4{]} + 1)
partitions\_matrix \textless{}- rbind(t\_c.partitions, m.partitions,
omega.partitions, phi.partitions)

\begin{verbatim}
# Searching procedure
while (iter_without_improvement < 100){
  # Drawing n_c * n_s points for looking for new solutions
  chosen_cells <- t(sapply(partitions, function(x) sample(1:x, size = n_c, replace = FALSE)))
  drawn_points <- sapply(1:nrow(partitions_matrix), function(row) sapply(chosen_cells[row,], 
                                                                         function(x) as.vector(sapply(x,function(y) 
                                                                           runif(n = n_s, min = partitions_matrix[row,y], max = partitions_matrix[row,y + 1])))))
  colnames(drawn_points) <- c('t_c', 'm', 'omega', 'phi')
  
  # Computing the value of loss function for new points, dropping points returning losses in a taboo region
  losses <- apply(drawn_points, 1, function(x) L(A = A, B = B, C = C, t_c = x[1], m = x[2], omega = x[3], phi = x[4], price = base_data$price, t = base_data$t))
  drawn_points <- cbind('loss' = losses, drawn_points)
  drawn_points <- drawn_points[complete.cases(drawn_points),,drop = FALSE] # dropping all points with non-defined loss function
  losses <- losses[complete.cases(losses)] # dropping points from losses, too
  non.taboo <- ifelse(losses < taboo_condition, TRUE, FALSE)
  drawn_points <- drawn_points[non.taboo,,drop = FALSE]
  
  # Picking the nontaboo point with the lowest move value - move value is defined as loss at step t+1 minus loss at step t
  if (nrow(drawn_points) > 0){
    S[['loss']] <- drawn_points[which.min(drawn_points[,'loss'] - S$loss), 'loss']
    S[['t_c']] <- drawn_points[which.min(drawn_points[,'loss'] - S$loss), 't_c']
    S[['m']] <- drawn_points[which.min(drawn_points[,'loss'] - S$loss), 'm']
    S[['omega']] <- drawn_points[which.min(drawn_points[,'loss'] - S$loss), 'omega']
    S[['phi']] <- drawn_points[which.min(drawn_points[,'loss'] - S$loss), 'phi']
    
    # Executing elite_list modification in case of the loss of a new points is lower than the loss of the 10th element in the elite_list
    if (S$loss < elite_list[10,'loss']){
      elite_list <- rbind(elite_list[1:9,], drawn_points[which.min(drawn_points[,'loss'] - S$loss),]) %>%
        arrange(loss)
      iter_without_improvement <- 0
    } else {
      iter_without_improvement <- iter_without_improvement + 1
    }
  } else {
    iter_without_improvement <- iter_without_improvement + 1
  }
}
\end{verbatim}

\} \# Filling results to the Grid grid \textless{}- grid
\%\textgreater{}\% filter(a == A, b == B, c == C) \%\textgreater{}\%
mutate(loss = elite\_list{[}1,`loss'{]}, t\_c =
elite\_list{[}1,`t\_c'{]}, m = elite\_list{[}1,`m'{]}, omega =
elite\_list{[}1,`omega'{]}, phi = elite\_list{[}1,`phi'{]})
print(elite\_list) return(grid) \}


\end{document}


\end{document}
